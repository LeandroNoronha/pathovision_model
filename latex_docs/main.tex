%
% Exemplo LaTeX de artigo UNISINOS
%
% Elaborado com base nas orientações dadas no documento
% ``GUIA PARA ELABORAÇÃO DE TRABALHOS ACADÊMICOS''
% disponível no site da biblioteca da Unisinos.
% http://www.unisinos.br/biblioteca
% 
% Os elementos textuais abaixo são apresentados na ordem em que devem
% aparecer no documento.  Repare que nem todos são obrigatórios - isso
% é devidamente indicado em cada caso.
%
% Comentários abaixo colocados entre aspas (`` '') foram
% extraídos diretamente do documento da biblioteca.
%
% Este documento é de domínio público.
%

%=======================================================================
% Declarações iniciais identificando a classe de documento e
% selecionando alguns pacotes adicionais.
%
% As opções disponíveis (separe-as com vírgulas, sem espaço) são:
% - twoside: Formata o documento para impressão frente-e-verso
%   (o default é somente-frente)
% - english,brazilian,french,german,etc.: idiomas usados no documento.
%   Deve ser colocado por último o idioma principal.
%=======================================================================
\documentclass[english,brazilian]{article_unisinos} %twoside
\usepackage[utf8]{inputenc} % charset do texto (utf8, latin1, etc.)
\usepackage[T1]{fontenc} % encoding da fonte (afeta a sep. de sílabas)
\usepackage{graphicx} % comandos para gráficos e inclusão de figuras
\usepackage{bibentry} % para inserir refs. bib. no meio do texto
\usepackage{float}
\usepackage{array}
\usepackage{booktabs}
\usepackage{multirow}
\usepackage{newfloat}
\usepackage{setspace}
%\usepackage{longtable}

\DeclareFloatingEnvironment[
fileext=loq,
listname={Lista de Quadros},
name=Quadro,
%placement=p,
%within=section,
%chapterlistsgaps=off
]{quadro}

\newcolumntype{L}{>{\centering\arraybackslash}m{6.2cm}}
\newcolumntype{G}{>{\centering\arraybackslash}m{5.5cm}}
\newcolumntype{A}{>{\centering\arraybackslash}m{2.4cm}}
\newcolumntype{B}{>{\centering\arraybackslash}m{7.5cm}}
%=======================================================================
% Escolha do sistema para geração de referências bibliográficas.
%
% O default é usar o estilo unisinos.bst.  Comente a definição abaixo
% e descomente a linha seguinte para usar o estilo do ABNTeX (é
% necessário ter esse pacote instalado).
%
% A vantagem do unisinos.bst é que ele permite o uso de um arquivo .bib
% seguindo as orientações tradicionais do BibTeX (veja essas orientações
% em http://ctan.tug.org/tex-archive/biblio/bibtex/contrib/doc/btxdoc.pdf).
% Entretanto, o estilo não suporta algumas citações mais exóticas como
% apud.  Para isso, use o ABNTeX, mas esteja ciente de que muitas de
% suas referências serão incompatíveis com os estilos tradicionais do
% BibTeX como plain, alpha, ieeetr, entre outros.
%=======================================================================
\usepackage[alf]{abntex2cite}

\autor{NORONHA DA SILVA}{LEANDRO}
\titulo{PATHOVISION: AI-Powered Image Analysis for Infectious Disease Detection}
\subtitulo{}
\orientador[Prof. Ph.D.]{Roehrs}{Alex}
%\coorientador[Prof.~Dr.]{Lamport}{Leslie}
\local{São Leopoldo}
\ano{2025}
\unidade{Academic Unit od Undergraduate}
\curso{Bachelor's Degree in Information Systems}
\natureza{%
Article presented as a partial requirement to obtain the Bachelor’s degree in Information Systems,
from the Information Systems Course at the University of Vale do Rio dos Sinos (UNISINOS)}

%=======================================================================
% Início do documento.
%=======================================================================

\begin{document}
\capa
\folhaderosto

% Diferentemente do normal, os comandos a seguir devem vir aqui mesmo,
% e não antes do \begin{document} onde seria o lugar deles. 
\tituloArtigo{PATHOVISION: AI-Powered Image Analysis for Infectious Disease Detection}{}
\autorArtigo{Leandro Noronha da Silva\footnote{Graduando em Sistemas de Informação pela Unisinos.  Email: leandrosilva6@edu.unisinos.br}}
\autorArtigo{Alex Roehrs\footnote{Mini-currículo do orientador. Email: usuario@unisinos.br}}

%=======================================================================
% Resumo em Português.
%
% A recomendação é para 150 a 250 palavras.
%=======================================================================
\begin{abstract}
O resumo, no idioma de escrita do artigo deve ser um texto único e com espaçamento \textbf{simples} entre linhas. Além disso, deve conter entre 100 a 250 palavras. Aqui é o abstract.

\palavraschave{Sugere-se de 3 a 5 palavras, separadas entre si por ponto final.}
\end{abstract}

\begin{otherlanguage}{english}
\singlespacing
\begin{abstract}
Resumo em inglês.
\end{abstract}
\palavraschave{Tradução das palavras-chave.}
\end{otherlanguage}

%=======================================================================
% Introdução
%=======================================================================
\section{Introdução Teste}

Neste documento é apresentado um exemplo de estrutura do artigo, podendo ser alterado de acordo com a combinação entre professor e aluno durante o processo de orientação.

Conforme \cite{artzi2020prediction} os registros de saúde são importantes.

De acodo com \cite{sterling2019prediction} a predição...

Neste trecho, segue um exemplo de referências a imagens presentes no texto, referenciando, neste caso, a Figura \ref{fig:logotipo}.

\begin{figure}[ht!]
	\caption{Ciência da Computação - Unisinos}
	\label{fig:logotipo}
	\centering%
	\begin{minipage}{.9\textwidth}
		\includegraphics[width=\textwidth]{images/ciencia-da-computacao.png}
		\fonte{\cite{computacaoUnisinos}}
	\end{minipage}
\end{figure}

De acordo com \cite{kumar2019traceability} blockchain para a ....


\section{Nova seção}
Texto da nova seção. Conforme a Figura \ref{fig:logotipo} podemos ver que tal coisa....

Conforme \cite{tait2021electronic} os registros de saúde são bons para....

Conforme \cite{charles2022health} o blockchain é \textbf{importante}....

\subsection{Subseção}
Texto da subseção, com exemplo de citação \cite{book}.

Conforme o autor \cite{nguyen2019closing} o UX Design está evoluindo de forma consistente em 2021.

Conforme \cite{febrero2021impact} a doação de órgão é tal coisa....

\subsection{Subseção}
Texto da subseção, com exemplo de citação \cite{article}.

Segundo \cite{vellido2020importance} a modelagem de dados na área da medicina....

De acordo com \cite{bogner2022type} o typescript...

\section{Nova seção}
Texto da subseção, com exemplo de citação \cite{inproceedings}.


De acordo com \cite{saiedian2000requirements} .....

\section{Conclusões e Trabalhos Futuros}

Gostei da opinião do \cite{biswas2023role}, por que....

\bibliography{exemplo}

\appendix
\section{NOME DO APÊNDICE}

Conteúdo do Apêndice. Não esqueça de referenciar os apêndices no corpo do texto.

Os apêndices são elementos opcionais e não contam para os números mínimo e máximo de páginas do regulamento do TCC da Unisinos (mínimo 25 e máximo 30).

\annex
\section{NOME DO ANEXO}
Conteúdo do Apêndice. Não esqueça de referenciar os anexos no corpo do texto.

Os anexos são elementos opcionais e não contam para os números mínimo e máximo de páginas do regulamento do TCC da Unisinos (mínimo 25 e máximo 30).

\end{document}
